This Ph.D. thesis represents almost four years of research, and many hours of writing, which would not have been possible without the help, guidance, advice and reassurance of many people. It is in this section that I thank many of them, and hope that the ones I omit realise that I do so purely out of forgetfulness.

I must first thank my supervisor, Yorck Ramachers, for providing the opportunity to make real progress in a field of particle physics that was severely under-resourced, as well as for his advice and guidance over the four years that have gone into the work presented here. The discussions, which ranged from short and direct to long and wandering, were nonetheless always interesting and useful, and I would not have made it to this point without his support.

In addition, I would like to thank Gary Barker and Steven Boyd for their unending questions, many of which helped to highlight major issues that must be resolved or explained in order to proceed. Their sharp eyes are responsible for noticing several issues that may otherwise have been left unexplored. I must also thank Ben Morgan for his efforts in supporting the research presented here, most notably by writing and maintaining the Geant4-based \emph{Lamu} simulation central to much of the analysis in this thesis.

Although not directly involved in my research, I would like to acknowledge the roles of Phill Litchfield and Tom Latham, both of whom provided advice, discussion and entertainment when needed. They are both extremely competent particle physicists, and my Ph.D. experience was enriched by the many discussions with them. I would like to thank my friends and colleagues, Leigh Whitehead, Mark Whitehead, Martin Haigh, Eugenia Puccio, Nicola McConkey, Callum Lister (who painstakingly proof-read much of this thesis!) and Daniel Scully for providing an office environment that never got dull, and for being the source of much insightful conversation about particle physics and the wider world. Finally, I would like to acknowledge the work of Ben Newell, a summer research student who laid the foundations for the cellular automaton that was later adapted and became a central point of my Ph.D., and who engaged thoroughly in the spirit of interesting conversation within the particle physics group.
