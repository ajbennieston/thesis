\chapter{Conclusion}\label{chapter:Conclusion}
A number of algorithms and reconstruction techniques have been introduced in this thesis, and it is worth taking some time to summarize the important results of the characterisation of each.

Chapter \ref{chapter:Latte} introduced a number of algorithms that can be used as building blocks for more complicated reconstruction procedures. In particular, the KDTree has been used to provide near-neighbour search in the cellular automaton of chapter \ref{chapter:CellularAutomaton} and in the length estimation algorithms of chapter \ref{chapter:Analysis}. The KDTree is not a reconstruction algorithm in itself, but is demonstrably useful in the implementation of such algorithms.

Of particular note in chapter \ref{chapter:Latte} is the three-dimensional feature detection, which was demonstrated to correctly identify the primary interaction vertex (as well as a number of other features such as the production sites of delta electrons, and track endpoints) in over $70\%$ of charged current $\nu_\mu$ interactions producing $\mu + p$ final states. The ability to identify these interest points with high efficiency and in three dimensions is a significant result. The technique could be further improved by extending the mathematical oeprations of the feature detection algorithm to three dimensions, rather than using projections as described in this thesis. Such an extension would take full advantage of the three-dimensional data and may result in more accurate determination of features of interest. However, this task relies on finding a suitable function to produce a scalar \emph{response value} from the response matrix.

The cylinder merging algorithm presented merges hits that fall inside an infinitely long cylinder around a seed track. This is a successful strategy for very sparse events, but with higher track multiplicities, it is possible to have intersecting cylinders from key tracks. In such a situation, stealing hits from one track is not desirable, so the algorithm could be trivially extended by considering only the regions near the end of a track. In events with a high track multiplicity, or with high curvature, such a merging strategy would produce better results than the infinite cylinder method used here.

In order to automate the application of these algorithms to perform reconstruction tasks, the \emph{Latte Control} framework was developed and described in chapter \ref{sec:LatteControl}. This allowed for the prototyping of reconstruction sequences, as well as automating reconstruction in a controllable and customisable way. This framework makes automated reconstruction of \ac{LAr TPC} events possible by simply chaining together correctly configured components, which can perform any task from the initial reconstruction steps through to complex statistical analyses.

Chapter \ref{chapter:CellularAutomaton} introduced a clustering technique based on a cellular automaton. The performance of this algorithm was characterised by applying it to simple two-track straight line events before running on long two-track events from charged current $\nu_\mu$ interctions. The final step was to apply the algorithm to the full set of hits from charged current $\nu_\mu$ interactions producing either $\mu + p$ or $\mu + p + \pi^+$ final states. The resulting clusters were typically of very high purity (over $90\%$) and although some hit loss reduced the hit-level efficiency, many of these hits could be recovered by returning to the raw data and applying a merging algorithm to fill in any dropped hits.

The cellular automaton itself produces small clusters of high purity based on its ability to discriminate between sets of related hits according to how track-like they are, and according to the directionality of those tracks. When applied to straight line two-track events with a large opening angle (i.e. over $140\degree$) this ability to discriminate is diminished; applying feature detection to isolate the vertex and mask out hits in that region may help to improve the clustering in these cases.

Chapter \ref{chapter:KalmanFilter} describes a Kalman filter for muon momentum reconstruction, as presented by the Icarus collaboration. An attempt to re-implement this filter and apply it to muon data from the \emph{Lamu} simulation proved unsuccessful, with no reliable momentum measurements available. It is possible that with further tuning, the Kalman filter may be able to provide realistic momentum estimates from measurements that take into account the multiple scattering of particles in a dense medium, but this will always be an easier task to perform in the presence of a magnetic field, where a helical fit can be applied. Given the large size of planned liquid Argon detectors, a strong magnetic field would be prohibitively expensive, but weaker, non-uniform fields may be sufficient to perform this task. A magnetic field would also provide much needed charge discrimination, e.g. between $\mu^+$ and $\mu^-$.
