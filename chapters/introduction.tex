\chapter{Introduction}
This thesis explores reconstruction procedures for use in a \ac{LAr TPC}. Fully automated reconstruction of neutrino events in these environments has not been successfully demonstrated previously, although several collaborations across the world are working towards this goal. This thesis discusses a number of algorithms and techniques, and assesses their applicability to the field of reconstruction in fine-grained detector environments.

These environments present a number of challenges, including multiple scattering occurring throughout the active detector volume. Furthermore, in neutrino experiments, the primary vertices of events are not well-localised to a single interaction point. This means that algorithms must exhibit some degree of translational invariance, i.e. they must be insensitive to the precise starting point of the event.

Many of the techniques presented existed as algorithms in other fields of science, mostly the machine vision and learning disciplines of computer science, while others are more traditional particle physics techniques that were adapted for the unique environment provided by \acs{LAr TPC}s. Each technique is described in terms of its theoretical or computational basis before the implementation details are explained and the results of applying the technique to \acs{LAr TPC} data are presented.

The ultimate goal is to present options for a fully automated reconstruction chain, going from detector readout at one end to physics information at the other. This chain should require no manual intervention\footnote{Many existing ``bubble chamber''-like experiments use some degree of human interaction to scan events and pick out starting points for less sophisticated computer algorithms.} and should be free of arbitrarily defined cuts or thresholds. In every case, where a parameter, cut value or threshold is required, it should be justified in terms of the data and the performance of the algorithm.

In addition to the algorithms and techniques themselves, this thesis also presents a framework known as \emph{Latte}, in which fully automated reconstruction and analysis tasks may be performed, making use of a pipeline architecture to simplify the flow of data from one algorithm to the next, and automating tasks such as reading event data from disk, looping over events, and writing results back to disk. The algorithms presented here all work in conjunction with this framework, and each algorithm implemented contains a control module for use with Latte.

The emphasis throughout this thesis is on small, self-contained algorithms that are verified against truth data to the maximum extent possible. These can be used as building blocks to establish a reconstruction chain that has known properties.

The structure of this thesis is to present the current status of neutrino physics in chapter \ref{chapter:NeutrinoPhysics} before explaining the relevance of \ac{LAr TPC}s to the global experimental neutrino programme in chapter \ref{chapter:DetectorPhysics}, as well as looking at the details of \ac{LAr TPC}s from a detector physics perspective. In chapter \ref{chapter:Latte}, the Latte framework is described in detail, along with a number of the smaller algorithms and services it provides. While these algorithms are grouped together here, they are in no way less important than those that are the subject of an entire chapter.

Chapter \ref{chapter:CellularAutomaton} describes a clustering algorithm which uses a cellular automaton to find track-like objects in heterogeneous event structures (i.e. those containing tracks and additional, unrelated hits from detector noise, co-developing showers, etc.), while chapter \ref{chapter:KalmanFilter} describes a Kalman filter designed for obtaining momentum estimates for muon tracks. Chapter \ref{chapter:PID} describes particle identification techniques of relevance for extracting muon tracks from a reconstructed event, and chapter \ref{chapter:Analysis} attempts to put these building blocks together into an end-to-end analysis.

Finally, conclusions are drawn in chapter \ref{chapter:Conclusion}, summarising the results from the other chapters.
