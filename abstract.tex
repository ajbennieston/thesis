The use of Liquid Argon Time-Projection Chambers in neutrino physics is looking increasingly certain as the field moves to larger, finer-grained detectors capable of delivering the physics reach required for the next generation of experiments studying neutrino oscillations and $\Charge\Parity$ violation in the lepton sector. 

This thesis explores reconstruction procedures for use in a \ac{LAr TPC}. Fully automated reconstruction of neutrino events in these environments has not been successfully demonstrated previously, although several collaborations across the world are working towards this goal. A number of algorithms and techniques are discussed, and their applicability to the field of reconstruction in fine-grained detector environments is assessed.

The techniques presented here are fully automated and characterised to the maximum extent possible, and may be combined to produce a software reconstruction chain that is free from human intervention. In addition to these algorithms, a framework for running chained reconstruction tasks is presented and demonstrated to work in conjunction with the algorithms developed.

Muon identification is also presented, using cuts justified from the truth information available from simulations. The algorithms and cuts together are used to analyse simulated neutrino events throughout the thesis, focusing on charged current muon neutrino interactions at energies of $0.77\GeV$ and $4.5\GeV$, and considering interactions producing $\ccqe$ (referred to as CCQE) or $\ccpi$ (referred to as $\CCPI$) final states.


